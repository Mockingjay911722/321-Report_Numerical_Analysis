\chapter{Introduction}

I divide my report into five parts. The first one is this the Introduction. Then I focus on how to find pair of eigenvalues and eigenvectors on Chapter \ref{chapter2}. In terms of Chapter \ref{chapter3}. It shows how to solve the question 2 on the coursework booklet. Later, Chapter \ref{chapter4} shows some of the questions that I encountered during writing part 2 and part 3, that I used the red color to make a comment. Meanwhile, the red comment is not only limited on offering a question but also shows some important connections to make my report flow. Finally, Chapter \ref{chapter5} is what I think I could utilize to my FYP, and my future research. This is also one of the most important reason that I chose this class, as it deeply connects with my interesting field, astrophysics. Meanwhile, due to the limited time, there were also some unsolved that I listed for future exploration.

In detail, for Chapter \ref{chapter2}, the most important algorithm QR algorithm came to the end of this chapter, with two sections: QR algorithm \ref{Section QR algorithm} and Continue QR \ref{Section Continue QR}. Could be clicked directly. Some mentioned may be not that important, but it holds some similar ideas for me to type to understand, also for history, maybe. All the materials, including the LaTeX file and the Matlab files, I upload to my GitHub for reference. In my GitHub, there were some of my past and ongoing projects as open sources, waiting for and welcoming to everyone's improvements and suggestions. The second problem is linked here for quick reference. Code \ref{Code} and Figure \ref{Figure 1}.

Answers to both questions were mostly based on \cite{arbenz2012lecture}. What I am doing in this report was to paraphrase, to write to understand.