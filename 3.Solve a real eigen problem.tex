\chapter{Solve a real eigenvalue problem }
\label{chapter3}

To solve the given eigenvalue problem using the finite difference method, we will:
\begin{enumerate}
    \item Discretize the differential equation using the finite difference method.
    \item Solve the resulting generalized eigenvalue problem.
    \item Identify negative eigenvalues and the smallest positive eigenvalue.
    \item Plot the corresponding eigenfunctions.
\end{enumerate}

Discretizing the domain is done by chopping the length of the 1D bar into chunks. That is, define discrete points on the bar
\[0 = x_0<x_1<x_1<...<x_n<x_{n+1} = 1\], which are referred to as gridpoints $x_i$. We let $u_i$ denote the numerical approximation of the exact solution $u(x_i)$ for $i = 0,1,...,n+1$
Here we assume evenly spaced intervals/ define the grid spacing h as
\[h = x_i - x_{i-1} = \frac{1}{n+1}\]
Note that due to the boundary conditions given $\mu_{(0)} = 1$ and $\mu_{(1)}=1$. Therefore, the unknowns we must solve for are n values, that is $u_1.u_2,...,u_n$(i.e. at non-boundary gridpoints). These gridpoints are referred to as the active gridpoints. 

Recall, finite differences are one approach to obtain discrete approximations of derivatives. For example,
\[\frac{\partial u}{\partial x} \approx \frac{u_{(x_i)}-u_{(x_{i-1})}}{x_i-x_{i-1}} \approx \frac{T_i - T_{i-1}}{h}\] 
And then by using the centred finite difference approximation
\[\frac{\partial^2 u}{\partial x^2} \approx \frac{\frac{T_{i+1-T_i}}{h}-\frac{T_i-T_{i-1}}{h}}{h}=\frac{T_{i+1}-2T_i+T_{i+1}}{h^2}\]

Specifically, let's look at this with a simplified version (only with $-\mu^{''}$ that: given each exact solution a equation relating to its value to its two neighbors:

\begin{equation}
    -\frac{u_{i+1}-2u_i+u_{i+1}}{h^2} = f_i \text{ for } i = 1,...,n
\end{equation}

\begin{equation}
    \frac{1}{h^2} \begin{bmatrix}
        2 & -1 \\
        -1 &2 &-1 \\
        &\ddots & \ddots & \ddots \\
        & & -1 & 2 & -1\\
        &&&-1&2\\
    \end{bmatrix} \begin{bmatrix}
        u_1 \\ u_2 \\ \vdots \\ u_{n-1} \\u_n
    \end{bmatrix} = \begin{bmatrix}
        f_1 \\ f_2 \\ \vdots \\ f_{n-1} \\f_n
    \end{bmatrix}
\end{equation}



Then the code is shown:
\label{Code}
\begin{verbatim}
% Parameters
N = 150;               % Number of interior points
h = 1 / (N + 1);       % Step size
x = linspace(0, 1, N + 2)'; % Grid points including boundaries

% Define a(x)
a = zeros(N, 1);
for i = 1:N
    xi = x(i+1); % Interior points
    if xi < 1/4 || xi > 3/4
        a(i) = 20;
    else
        a(i) = 10;
    end
end

% Construct the finite difference matrix
A = zeros(N, N);
for i = 1:N
    if i > 1
        A(i, i-1) = -1 / h^2;   % Coefficient for u_{i-1}
    end
    A(i, i) = 2 / h^2 + a(i); % Coefficient for u_i
    if i < N
        A(i, i+1) = -1 / h^2;   % Coefficient for u_{i+1}
    end
end

% Solve the eigenvalue problem
[eigenvectors, eigenvalues] = eig(A);
lambda = diag(eigenvalues); % Extract eigenvalues

% Sort eigenvalues
[lambda_sorted, idx] = sort(lambda);
eigenvectors_sorted = eigenvectors(:, idx);

% Extract negative eigenvalues and smallest positive eigenvalue
negative_eigenvalues = lambda_sorted(lambda_sorted < 0);
smallest_positive_eigenvalue = min(lambda_sorted(lambda_sorted > 0));

% Display results with 6 significant figures
disp('Negative Eigenvalues (6 significant figures):');
fprintf('%.6f\n', negative_eigenvalues);

disp('Smallest Positive Eigenvalue (6 significant figures):');
fprintf('%.6f\n', smallest_positive_eigenvalue);

% Plot first few eigenfunctions
figure;
hold on;
for k = 1:min(3, length(lambda_sorted)) % Plot first 3 eigenfunctions
    u_k = [0; eigenvectors_sorted(:, k); 0]; % Add boundary conditions
    plot(x, u_k, 'DisplayName', ['\lambda = ', num2str(lambda_sorted(k))]);
end
hold off;
xlabel('x');
ylabel('u(x)');
title('Eigenfunctions');
legend show;
grid on;
\end{verbatim}
And then the results:

Firstly:
\begin{verbatim}
>> model3
Negative Eigenvalues (6 significant figures):
-2.146894
Smallest Positive Eigenvalue (6 significant figures):
24.208290
>> 
\end{verbatim}
Then the plot:
\begin{figure}
    \centering
    \label{Figure 1}
    \includegraphics[width=1.0\columnwidth]{Image/figure1.jpg}
    \caption[Eigenvectors of the corresponding eigenvalues]{Eigenvectors of the corresponding eigenvalues}
    \label{}
\end{figure}





